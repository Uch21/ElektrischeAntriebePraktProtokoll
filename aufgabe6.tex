\chapter{}\label{ex:aufg6}
%
\section{}\label{sec:aufg6a}
%
Der Unterschied zwischen dem idealen und dem realen Stromverlauf ist der, dass ideal eine Rechteckform entsteht, real kann der Strom an der Spule, aufgrund des ohmschen Widerstandes, nicht springen, er verläuft näherungsweise nach einer Exponentialfunktion beim Einschalten der Phase. Der Motor besteht aus einer Induktivität und einem Widerstand, diese legen die Zeitkonstante $\tau$ fest, mit der die Exponentialfunktion ansteigt. Wenn nun der Strom kommutiert, das heißt der Strom fließt über die jeweils andere Phase, erfährt dieser einen Einbruch,  weil in der anderen Wicklung derselbe Effekt auftritt.
%
\section{}\label{sec:aufg6b}
%
Aufgrund dessen, dass das Drehmoment durch den Strom erzeugt wird, bricht es im gleichen Maße ein, wie der Strom.
%
\section{}\label{sec:aufg6c}
%
Da es nicht klug ist, die Induktivität oder den ohmschen Widerstand des Motors zu verändern, um eine geringer Zeitkonstante zu erreichen, kann das Verhalten durch einen Stromregler verbessert werden.