\chapter{}\label{ex:aufg4}
%
\section{}\label{sec:aufg4a}
Da die Drehzahl $n_s$ einer Asynchronmaschine die Abhängigkeit
\begin{equation}
	n_s = \frac{f}{Z_P}\cdot \left(\frac{60~\text{s}}{\text{min}}\right)
\end{equation}
der Netzfrequenz $f$ und der Polpaarzahl $Z_P$  hat, ergibt sich daraus für $Z_P$
\begin{equation}
	Z_P = \frac{f}{n_s}\cdot \left(\frac{60~\text{s}}{\text{min}}\right)
\end{equation}
%

\section{}\label{sec:aufg4b}
Die Asynchronmaschine im Versuch wird mit $f = 50$ Hz betrieben und hat eine Nenndrehzahl $n_s = 1370 \mathrm{min}^{-1}$, somit ergibt sich eine Polpaarzahl von
\begin{equation}
	Z_P = \frac{50~\text{Hz}}{1370~\text{min}^{-1}} \cdot \left(\frac{60~\text{s}}{\text{min}}\right) \approx 2
\end{equation}