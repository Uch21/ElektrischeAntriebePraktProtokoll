\chapter{}\label{ex:aufg4}
%
\section{}\label{sec:aufg4a}
Die hier dargestellte Gleichung (\ref{eq:Rotorkf}) entspricht der Gleichung (6.49) aus dem Skript Elektrische Antriebe und gibt die Beziehung zwischen mechanischer Rotorkreisfrequenz $\Omega_{Rm} = 2\pi \cdot N$, Statorkreisfrequenz $\Omega_1$, Polpaarzahl $Z_P$ und Schlupf $s$ an.
\begin{equation}
	\Omega_{Rm} = \frac{\Omega_1}{Z_P}\cdot (1-s)
	\label{eq:Rotorkf}
\end{equation}
Beim Betrieb der Asynchronmaschine am starren Netz entspricht die Statorfrequenz $f_1 = \frac{\Omega_1}{2\pi}$ der Netzfrequenz. Außerdem gilt die Beziehung $s = \frac{f_2}{f_1}$, wobei $f_2$ die Schlupffrequenz ist. Somit ergibt sich
\begin{equation}
	N = \frac{f_1 - f_2}{Z_P}
	\label{eq:Drehzahl}
\end{equation}
Da außer der Polpaarzahl auch die Schlupffrequenz unbekannt ist, müssen verschiedene Werte für $Z_P$ ausprobiert werden. Mithilfe des sich ergebenden Wertes für die Schlupffrequenz kann eine Plausibilitätsbetrachtung durchgeführtund die richtige Polpaarzahl bestimmt werden. Dazu können die Beziehungen
\begin{equation}
    N_0 = \frac{f_1}{Z_P}
\end{equation}
\begin{equation}
    N_2 = \frac{f_2}{Z_P} = N_N - N_0
\end{equation}
verwendet werden. Dabei ist $N_0$ die Leerlaufdrehzahl, $N_N$ die auf dem Typenschild angegebene Nenndrehzahl und $N_2$ die Schlupfdrehzahl.

\section{}\label{sec:aufg4b}
Die Asynchronmaschine im Versuch wird mit $f_1 = 50~\text{Hz} = 3000~\text{min}^{-1}$ betrieben und hat eine Nenndrehzahl $N_N = 1370~\text{min}^{-1}$. Mit diesen Angaben kann nach Teilaufgabe \ref{sec:aufg4a}) die Polpaarzahl bestimmt werden:
\begin{description}
    \item[$Z_P = 1$:] Mit $Z_P = 1$ ergibt sich $N_0 = 3000~\text{min}^{-1}$ und damit $N_2 = 1630~\text{min}^{-1}$. Diese Schlupfdrehzahl ist viel zu hoch. Deshalb kann $Z_P = 1$ ausgeschlossen werden.
    \item[$Z_P = 2$:] Man erhält die Leerlaufdrehzahl $N_0 = 1500~\text{min}^{-1}$. Somit würde die Schlupfdrehzahl $N_2 = 130~\text{min}^{-1}$ betragen. Dieser Wert ist plausibel.
\end{description}