\chapter{}\label{ex:aufg5}
%
\section{}\label{sec:aufg5a}
%
Bei $M_L = 0$ Nm bekommen wir druch umstellen der Formel (4.1) nach der Frequenz als Drehzahlen $n_1 = 600 \text{min}^-1$, $n_2 = 1500 \text{min}^-1$ und $n_3 = 2400 \text{min}^-1$.
\section{}\label{sec:aufg5b}
%
Nach dem Einstellen der Frequenz erhalten wir aus der Messung folgende Drehzahlen:\\
$n_1 = 583 \text{min}^-1$, $n_2 = 1479 \text{min}^-1$ und $n_3 = 2354 \text{min}^-1$.
Der Grund für die Abweichung liegt darin, dass wir keine idealen Bauelemente haben und ein Lastmoment von 0 Nm nie ganz erreicht werden kann aufgrund der Lagern und des Lüfters.

$\begin{array}{c | c}
	\text{Berechnete}~n_s/\text{min}^-1 & \text{Gemessene} ~n_s/\text{min}^-1\\
	 600 &  583 \\ 
	1500 & 1479 \\ 
	2400 & 2354
\end{array} $