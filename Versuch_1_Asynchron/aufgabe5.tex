\chapter{}\label{ex:aufg5}
%
\section{}\label{sec:aufg5a}
%
Bei $M_L = 0$ Nm erhält man durch Umstellen der Formel \ref{eq:Drehzahl} für die Statorfrequenz die Werte $f_1 = 20 \text{Hz}$, $f_2 = 50 \text{Hz}$ und $f_3 = 80 \text{Hz}$.
\section{}\label{sec:aufg5b}
%
Nach dem Einstellen der Frequenz erhalten wir aus der Messung folgende Drehzahlen:\\
$N_1 = 583 \text{min}^{-1}$, $N_2 = 1479 \text{min}^{-1}$ und $N_3 = 2354 \text{min}^{-1}$.\\
Der Grund für die Abweichung liegt darin, dass wir keine idealen Bauelemente haben und ein Lastmoment von 0 Nm nie ganz erreicht werden kann aufgrund der Lager und des Lüfters. Die gewünschten und gemessenen Werte sind nochmals in Tabelle \ref{tab:drehzahlen} dargestellt.
\begin{table}[htb]
	\centering
	\begin{tabular}{c | c}
		Gewünschte $N/\text{min}^{-1}$ & Gemessene $N/\text{min}^{-1}$\\\hline
		600 &  583 \\ 
		1500 & 1479 \\ 
		2400 & 2354
	\end{tabular} 
	\caption{Gegenüberstellung der gewünschten und gemessenen Drehzahlen}
	\label{tab:drehzahlen}
\end{table}
	
	
