\chapter{}\label{ex:aufg5}

\section{}\label{sec:aufg5a}
Die Formel für den Winkel eines Vollschritts bzw. eine Halbschritts sind
\begin{equation}
	\alpha_{VS} = \frac{360^\circ}{2Z_P m}
\end{equation}
\begin{equation}
\alpha_{HS} = \frac{360^\circ}{4Z_P m}
\end{equation}
Somit ergibt sich für $\alpha_{VS} = 90^\circ$ und für $\alpha_{HS} = 45^\circ$ als Schrittwinkel für die im Skript in Abb. 7.1 dargestellten Schrittmotor.

\section{}\label{sec:aufg5b}
Mit den selben Formeln (5.1) und(5.2) können für den Schrittmotor mit Poolparzahl $Z_P = 50$ und der Strangzahl $m = 2$ ebenfalls die Schrittwinkel berechnet werden. Die sind $\alpha_{VS} =1.8^\circ$ und für $\alpha_{HS} = 0.9^\circ$.
Für die Schrittzahl $S$ pro Umdrehung $z$ wird folgende Formel verwendet.
\begin{equation}		
	1\frac{S}{z} = \frac{360^\circ}{\alpha}
\end{equation}
Für den Vollschritt wird die Formel (5.1) wird in (5.3) eingesetzt
\begin{equation}
	1\frac{S}{z} = 2\cdot Z_P m = 200 \frac{S}{z}
\end{equation}
und für den Halbschritt wird die Formel (5.2) in (5.3) eingesetzt
\begin{equation}
	1\frac{S}{z} = 4\cdot Z_P m = 400 \frac{S}{z}
\end{equation}

%
\section{}\label{sec:aufg5c}
Im Vollschrittbetrieb muss der Schrittmotor 240 Schritte ausführen um den Schlitten 24mm zu verfahren. Im Halbschrittbetrieb müssen doppelt so viele Schritte ausgeführt werden.
\clearpage