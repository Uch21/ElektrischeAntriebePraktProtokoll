\chapter{}\label{ex:aufg5}

\section{}\label{sec:aufg5a}
Mit der Formel für den Vollschritt
\begin{equation}
	\alpha_{VS} = \frac{360^\circ}{2\cdot Z_Pm}
\end{equation}
und die Formel für den Halbschritt
\begin{equation}
	\alpha_{HS} = \frac{360^\circ}{4\cdot Z_Pm}
\end{equation}
Ergibt einen Schrittwinkel von $\alpha_{VS} = 90^\circ$ und $\alpha_{HS} = 45^\circ$.
\section{}\label{sec:aufg5b}
Wenn man als Poolpaarzahl $Z_P = 50$ und Strangzahl $m = 2$ nimmt, ergeben sich die Schrittwinkel $\alpha_{VS} = 1.8^\circ$ und $\alpha_{HS} = 0.9^\circ$.
Schrittzahl pro Umdrehung ist beim Vollschritt 200 Schritte und beim Halbschritt 400.
\section{}\label{sec:aufg5c}
Vollschritt:
\begin{equation}
	\frac{20\frac{\mathrm{mm}}{\mathrm{U}}}{200\frac{\mathrm{Schritt}}{\mathrm{U}}} = 0.1 \frac{\mathrm{mm}}{\mathrm{Schritt}}
\end{equation}
\begin{equation}
	\frac{24 \mathrm{mm}}{0.1 \frac{\mathrm{mm}}{\mathrm{Schritt}}} = 240~\mathrm{Schritt}
\end{equation}
Halbschritt:
\begin{equation}
\frac{20\frac{\mathrm{mm}}{\mathrm{U}}}{400\frac{\mathrm{Schritt}}{\mathrm{U}}} = 0.05 \frac{\mathrm{mm}}{\mathrm{Schritt}}
\end{equation}
\begin{equation}
\frac{24 \mathrm{mm}}{0.05 \frac{\mathrm{mm}}{\mathrm{Schritt}}} = 480~\mathrm{Schritt}
\end{equation}
\clearpage