\chapter{}\label{ex:aufg6}
%
\section{}\label{sec:aufg6a}
%
Der Unterschied zwischen dem idealen und dem realen Stromverlauf ist der, dass ideal eine Rechteckform entsteht, real kann der Strom an der Spule, aufgrund der Reihenschaltung des ohmschen Widerstandes und der Induktivität, nicht springen, er verläuft näherungsweise nach einer Exponentialfunktion beim Einschalten der Phase (PT1-Verhalten). Wenn nun der Strom kommutiert, das heißt der Strom fließt über die jeweils andere Phase, erfährt dieser einen Einbruch,  weil in der anderen Wicklung derselbe Effekt auftritt.
%
\section{}\label{sec:aufg6b}
%
Aufgrund dessen, dass das Drehmoment durch den Strom erzeugt wird, bricht es im gleichen Maße ein, wie der Strom.
%
\section{}\label{sec:aufg6c}
%
Die einzige Möglichkeit, das beobachtete Verhalten zu verbessern, ist der Einsatz eines Stromreglers. In der Praxis wird zu diesem Zweck meist ein PI-Regler verwendet.