\chapter{}\label{ex:aufg3}
%
\section{}\label{sec:aufg3a}
Mit Gleichung (5.12) aus dem Skript elektrische Antriebe kann $c_\text{E} \Psi_\text{N}$ berechnet werden. Denn im Leerlauf ist $M_{Mi} = 0$, daraus folgt
\begin{equation}
c_\text{E} \Psi_\text{N} = \frac{U_{\text{AN}}}{N_{\text{N}0}}
\end{equation} 
Mit Gleichung (5.6) und (5.11) aus dem Skript elektrische Antriebe erhält man
\begin{equation}
R_A = \frac{U_{\text{AN}} - c_\text{E} \Psi_\text{N} \cdot N_\text{N}}{I_{\text{AN}}}
\end{equation} 

\section{}\label{sec:aufg3b}
%
%$U_A = 179.2\text{V}, N_{N0} = 1792min^-1$
Für $c_\text{E} \Psi_\text{N}$ ergab sich $c_E \Psi_N = 6.016 \text{Vs}$.
Die Berechnung von $R_\text{A}$ wurde erst mit den Werten des Typenschilds durchgeführt.
\begin{equation}
R_\text{A} = \frac{U_{\text{AN}} - c_\text{E} \Psi_\text{N} \cdot N_\text{N}}{I_{\text{AN}}} = \frac{210 \text{V} - 6.016\text{Vs} \cdot \frac{1900}{60\text{s}}}{1.61\text{A}} = 12.1078 \Omega
\end{equation}
Nach der Messung der Werte U, I und N ergab sich für $R_\text{A} = 8.12\Omega$.
%$U_A = 100 \text{V}, I_A = 0 \text{A}, N_0 = 1040 \frac{1}{\text{min}}, I_{AN} = 1.61\text{A}$

\clearpage