\chapter{}\label{ex:aufg3}
%
\section{}\label{sec:aufg3a}
Mit Gleichung (5.12) kann $c_E \Psi_N$ berechnet werden. Denn im Leerlauf ist $M_{Mi} = 0$, daraus folgt
\begin{equation}
c_E \Psi_N = \frac{N_{N0}}{U_{AN}}
\end{equation} 
Mit Gleichung (5.6) und (5.11) erhalt man
\begin{equation}
R_A = \frac{U_{AN} - c_E \Psi_N \cdot N_N}{I_{AN}}
\end{equation} 

\section{}\label{sec:aufg3b}
%
\clearpage